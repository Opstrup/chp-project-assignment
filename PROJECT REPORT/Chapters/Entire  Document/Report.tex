%%%%%%%%%%%%%%%%%%%%%%%%%%%%%%%%%%%%%%%%%%%%%%%%%%%%%%%%%%%%%%%%%%%%%%
% Project Report 
%%%%%%%%%%%%%%%%%%%%%%%%%%%%%%%%%%%%%%%%%%%%%%%%%%%%%%%%%%%%%%%%%%%%%%


%%% Preamble
\documentclass[paper=a4, fontsize=11pt]{scrartcl}
\usepackage[T1]{fontenc}
\usepackage{fourier}

\usepackage[english]{babel}															% English language/hyphenation
\usepackage[protrusion=true,expansion=true]{microtype}	
\usepackage{amsmath,amsfonts,amsthm} % Math packages
\usepackage[pdftex]{graphicx}	
\usepackage{url}


%%% Custom sectioning
\usepackage{sectsty}
\allsectionsfont{\centering \normalfont\scshape}


%%% Custom headers/footers (fancyhdr package)
\usepackage{fancyhdr}
\pagestyle{fancyplain}
\fancyhead{}											% No page header
\fancyfoot[L]{}											% Empty 
\fancyfoot[C]{}											% Empty
\fancyfoot[R]{\thepage}									% Pagenumbering
\renewcommand{\headrulewidth}{0pt}			% Remove header underlines
\renewcommand{\footrulewidth}{0pt}				% Remove footer underlines
\setlength{\headheight}{13.6pt}


%%% Equation and float numbering
\numberwithin{equation}{section}		% Equationnumbering: section.eq#
\numberwithin{figure}{section}			% Figurenumbering: section.fig#
\numberwithin{table}{section}				% Tablenumbering: section.tab#


%%% Maketitle metadata
\newcommand{\horrule}[1]{\rule{\linewidth}{#1}} 	% Horizontal rule

\title{
		%\vspace{-1in} 	
		\usefont{OT1}{bch}{b}{n}
		\normalfont \normalsize \textsc{Technical University of Denmark - DTU Compute -} \\ [20pt]
		\horrule{0.5pt} \\[0.4cm]
		\LARGE \textsc{[SuperStringWithExpansion]} Problem \\
		\horrule{2pt} \\[0.5cm]
}
\author{
		\normalfont 							
        Salik Lennert Pedersen \\
        Anders Holmgaard Opstrup \\
        Federico Bergamin \\ [-3pt]		\normalsize
}
\date{}



%%% Begin document
\begin{document}
\maketitle
\section{Description of the problem}
In this decisional problem we have two different alphabet $\mathcal{\Sigma}$ and $\mathcal{\Gamma}$, where $\mathcal{\Gamma}=\{\gamma_1,...,\gamma_m\}$ contains $m$ symbol, a string $s$ which is made up of symbols that belongs to the  $\mathcal{\Sigma}$ alphabet (formally $s\in\mathcal{\Sigma^*})$, $k$ strings $t_1,...,t_k$, which is made up of symbols from both alphabets, and in the end, we have $m$ subsets $R_1,...,R_m \subseteq \mathcal{\Sigma^*}$. These subsets contain symbols of $\mathcal{\Sigma}$ alphabet, and as we can notice, they are as many as the symbols in the alphabet $\mathcal{\Gamma}$. That's because for each $\gamma_i$ there is a subset $R_i$. We can understand better this relation explaining the core of our problem. \newline
\noindent The output of the problem will be YES if exits a sequence of words $r_1\in R_1$, $r_2\in R_2$,...,$r_m\in R_m$, such that for all $1 \leq i \leq k$ the so-called \textit{expansion} $e(t_i)$ is substring of $s$; the expansion of a string $t_i$ consists in the replacement of every symbols $\gamma_j \in \mathcal{\Gamma}$ that appears in the string $t_i$ with its expansion, where the expansion of a symbol is defined as follow: $e(\gamma_j):=r_j$. That is, we have to choose for every symbol $\gamma_j\in \mathcal{\Gamma}$ a symbol $r_j\in R_j$, and we know that $r_j \in\mathcal{\Sigma^*}$. We have to replace the symbol $\gamma_j$ in each string $t_i$ with the symbol $r_j$ that was chosen. We have to do that for each $\gamma \in \mathcal{\Gamma}$. In this way, using these replacements, we are transforming our string $t_i$ in its expansion $e(t_i)$. \newline 
In the end we will obtain that every expansion-string is made up of symbols of the alphabet $\mathcal{\Sigma}$ as our string $s$. Hence, if every new constructed string $e(t_i)$ is a substring of $s$, the answer for our problem will be YES, otherwise NO.
\newline
In other words, the answer of our problem will be YES if there is a selection of the $r_i \in R_i$ such that, replacing each symbols $\gamma_i \in \Gamma$ with the respective symbol $r_i$ (we have to remember that for each $\gamma_i$ we have to chose one symbol from the subset $R_i \subseteq \Sigma^*$) in every string $t\in T$, where $|T|=k$, we will have that every new string that we obtain with this replacement is a substring of the string $s$. 
If a selection of $r_i \in R_i$ like this doesn't exist, the answer will be NO.

\subsection{Simple examples}
With the text of the problem we receive also some problem instances on the alphabets $\Sigma=\texttt{\{a,b,...,z\}}$, $\Gamma=\texttt{\{A,B,...,Z\}}$. The first line of the file contains the number k, which is the number of strigs $t$ we have. The second line contains the string $s$ and the following $k$ lines the strings $t_1,..., t_k$. Finally, the last lines (at most 26, the number of letters in the alphabet) start with a letter $\gamma_j\in \Gamma$ followed by a colon and the contents of the set $R_j$ belonging to the letter, where the elements of the set are separated by commas. Hence, we can clearly see that for each symbols $\gamma_i \in \Gamma$, in this case letter, there is a subset $R_i$ and in this subset we have to chose one symbol.
\newline
\\
\noindent For example: \newline
\\
\noindent\texttt{4} \\
\noindent \texttt{abdde} \\
\noindent \texttt{ABD} \\
\noindent \texttt{DDE} \\
\noindent \texttt{AAB} \\
\noindent \texttt{ABd} \\
\noindent \texttt{A:a,b,c,d,e,f,dd} \\
\noindent \texttt{B:a,b,c,d,e,f,dd} \\
\noindent \texttt{C:a,b,c,d,e,f,dd} \\
\noindent \texttt{D:a,b,c,d,e,f,dd} \\
\noindent \texttt{E:aa,bd,c,d,e} \\

\noindent In this example we can observe that there is NO possible selection of symbols $r_i$ such that every string $t$ would be a substring of \texttt{s=abdde}. In fact if we have a look to $t_2$=\texttt{DDE} and $t_3$=\texttt{AAB}, we can notice that they have the same structure, and in the same time in our string $s$ the only letter that is repeated twice is the \texttt{d}. So we conclude that \texttt{A=d} and \texttt{D=d}. Moreover from $t_4$=\texttt{ABd} we see that, since $t_4$ will be a substring, or \texttt{B=d} or \texttt{B=b}. If we try to choose our selection of $r_i$ following these rules, we can noticed that there is no possible solution, such that at the same time, all $e(t_i)$ are substring of $s$.

\section{\textsc{SuperStringWithExpansion} is in $\mathcal{NP}$}
We have to show and prove that our problem is in $\mathcal{NP}$. To do that we use the so called \textit{guess-verify algorithms} proof, so we have to start designing a deterministic algorithm A which takes as input a problem instance X and a random sequence R.

\begin{proof}
	We have to show that there is a polynomial \textit{p} and a randomized \textit{p}-bounded algorithm $A$ which satisfies the condition of the class $\mathcal{NP}$. 
	


\end{proof}
%%% End document
\end{document}